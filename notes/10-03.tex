\documentclass[../main.tex]{subfiles}
\graphicspath{{\subfix{../images/}}}
\begin{document}



\section{Struktura pevných látek}
\subsection{Prvky symetrie}

\begin{enumerate}
    \item Inverze - bodová symetrie
    \item Osa souměrnosti
    \item Rotace
    \item Rovina souměrnosti
\end{enumerate}


\subsection{Složené prvky souměrnosti}

\begin{itemize}
    \item Rotačně inverzní osy - rotace kolem osy a inverze
    \item Rotačně reflexní osy - rotace kolem osy a zrcadlení
\end{itemize}


\subsection{Bodové grupy}
Prvky souměrnosti které při operacích zachovávají alespoň 1 bod prostoru nepohyblivý nazýváme bodovými grupami. 

Celkem existuje 32 bodových grup. 

Lze rozdělit do 7 skupin, tzv. krystalových soustav. 

Existuje popis např 6mm nebo $4\bar{3}m$. \todo{Bude na zkoušce k interpretaci}. 
Číslo říká četnost osy, m říká rovinu souměrnosti. 

\subsection{Bravaisova mřížka}
Pro volbu elementárního prvku platí pár pravidel které jsou v prezentaci.

Mřížka + báze spolu tvoří strukturu. 


\subsection{Značení uzlových bodů, přímek, rovin}

\begin{itemize}
    \item Bod = [[u v w]]
    \item Směr = [u v w]
    \item Rovina = (h k l)
    \item Krystalově ekvivalentní roviny = \{h k l\}
    \item Krystalově ekvivalentní směry = $<$u v w$>$
\end{itemize}

\subsection{Příklady}

\subsubsection{Směry}

Máme krychli, její souřadnice vrcholů jsou 000 až 111.

Pro šipku z 001 do 110 máme lehce $110-001 = 11(-1)$ to ale značíme 
jako $11\bar{1}$ protože nechceme -, celkově $[11\bar{1}]$.


Pro šipku ze středu strany mezi 101 a 100 do 111 máme $111 - 10\frac{1}{2} = 0 1 \frac{1}{2}$ ale 
nechceme zlomky tak značíme $[021]$


\subsubsection{Roviny}
Máme krychli abc. 

Rovina prochází body mimo krychli, 2a 2b 1c.

Poté provádíme inverzi a máme $\frac{1}{2}\frac{1}{2}\frac{1}{1}$, nechceme zlomky tak vynásobíme 2.

Celkem $(112)$

\subsection{Translační prvky souměrnosti}
\begin{itemize}
    \item Šroubové osy
    \item Roviny skluzu
\end{itemize}



\subsection{Prostorové grupy}

Celkem 230 grup. 



\end{document}